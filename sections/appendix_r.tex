\begin{appendices}
\chapter{Cercetări experimentale}

\section{Funcții generale}

Următorul cod a fost utilizat pentru a crea o serie de vizualizări, destinate unei înțelegeri îmbunatățite fața de diverse moduri de asignare a experimentelor. A fost utilizat limbajul \textit{R}, datorită suportului extinsiv al acestui pentru construcția diagramelor.

\begin{center}
	\begin{lstlisting}[language=r]
library(plyr)
library(ggplot2)
library(scales)
library(IDPmisc)
library(gridExtra)
library(cowplot)

x <- seq(0.01, 1, 0.01)
y <- seq(0.01, 1, 0.01)
data <- expand.grid(x, y)
colnames(data) <- c("p1", "p2")
head(data)

plotData <- function(d, title, name, 
		low_col = "red",
		mid_col = "#eeeeee", 
		high_col = "#66cc00") {
	x <- data.frame(p1 = data$p1, p2 = data$p2, n = d)
	ggplot(x, aes(p1, p2, z = n)) +  
		stat_summary_hex() + 
		scale_fill_gradient2(
			name = name,
			high = high_col,
			mid = mid_col,
			low = low_col
		) +
		labs(x = "Control", y = "Test") +
		ggtitle(title) +
		geom_abline(
			intercept = 0.10, 
			slope = 1, 
			colour='#666666') +
		geom_abline(
			intercept = -0.10, 
			slope = 1, 
			colour= '#666666') +
		theme(
			panel.background = element_blank(),
			plot.background = element_blank()
		)	 
}


ab.epsilonFn <- function(p1, p2) {
	0.5
}

epsilon.epsilonFn <- function(p1, p2) {
	if (p1 > p2) 0.95 else 0.05
}

prob.epsilonFn <- function(p1, p2) {
	p1 / (p1 + p2)
}

hybrid.epsilonFn <- function(p1, p2) {
	epsExploit <- if (p1 >= p2) 1.0 else 0.0
	epsExplore <- 0.5
	epsProb <- p1 / (p1 + p2)
	alpha <- 0.1
	beta <- 0.1
	eps <- alpha * epsExploit + 
	       beta * epsExplore + 
	       (1 - (alpha + beta)) * epsProb
	eps
}

	\end{lstlisting}
\end{center}

\break

\section{Analiza convergenței}

Folosind funcțiile prezentate anterior, am calculat durata de convergența pentru fiecare tip de abordare.

\begin{center}
	\begin{lstlisting}[language=r]

computeConvergence <- function(eps, p1, p2) {
	a <- (1.66 ^ 2)  * 
	       (eps*p1*(1 - p1) + (1 - eps)* p2*(1 - p2))
	b <- (abs(p1 - p2)^2) * eps * (1 - eps)
	a / b
}

evaluateConvergence <- function(epsilon) {
	n <- mdply(data, function(p1, p2) {
		eps <- epsilon(p1, p2)
		c <- computeConvergence(eps, p1, p2)
		if (is.infinite(c)) NA else c
	})
	names(n) <- c('p1', 'p2', 'n')
	n
}
	
ab.convergence <- evaluateConvergence(ab.epsilonFn)
epsilon.convergence <- evaluateConvergence(epsilon.epsilonFn)
prob.convergence <- evaluateConvergence(prob.epsilonFn)
hybrid.convergence <- evaluateConvergence(hybrid.epsilonFn)
	\end{lstlisting}
\end{center}

\section{Analiza regretului}

Folosind funcțiile prezentate anterior, am calculat regretul pentru fiecare tip de abordare.

\begin{center}
	\begin{lstlisting}[language=r]
evaluateRegret <- function(epsilon, name) {
	r <- mdply(data, function(p1, p2) {
	eps <- epsilon(p1, p2)
	if (p1 < p2) 
		eps * (p2 - p1) 
	else 
		(1 - eps) * abs(p1 - p2)
	})
	names(r) <- c('p1', 'p2', 'r')
	r
}

ab.regret <- evaluateRegret(ab.epsilonFn)
epsilon.regret <- evaluateRegret(epsilon.epsilonFn)
prob.regret <- evaluateRegret(prob.epsilonFn)
hybrid.regret <- evaluateRegret(hybrid.epsilonFn)

	\end{lstlisting}
\end{center}

\end{appendices}

