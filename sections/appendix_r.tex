\begin{appendices}
\chapter{Cercetări experimentale}

\section{Analiza timpului de convergență}

Următorul cod a fost utilizat pentru a crea o serie de vizualizări, destinate unei înțelegeri îmbunatățite fața de diverse moduri de asignare a experimentelor.

\begin{center}
	\begin{lstlisting}[language=r]
library(plyr)
library(ggplot2)
library(scales)

x <- seq(0.01, 1, 0.01)
y <- seq(0.01, 1, 0.01)
data <- expand.grid(x, y)
colnames(data) <- c("p1", "p2")
head(data)

plotConvergence <- function(data, title) {
	ggplot(data, aes(p1, p2, z = n)) +  
		stat_contour() + 
		geom_tile(aes(fill = n)) + 
		scale_fill_gradient(
			name = "Timp convergenta",
			limits = c(0, 10000),
			na.value = '#ff3333',
			low = "#ffffff",
			high = "#ff3333"
		) +
		labs(x = "Grup 1", y = "Grup 2") +
		ggtitle(title)
}

computeConvergence <- function(eps, p1, p2) {
	a <- (1.66 ^ 2)  * 
		( eps * p1 * (1 - p1) + (1 - eps) *  p2 * (1 - p2))
	b <- (abs(p1 - p2)^2) * eps * (1 - eps)
	a / b
}

evaluateConvergence <- function(epsilon, name) {
	n <- mdply(data, function(p1, p2) {
		eps <- epsilon(p1, p2)
		c <- computeConvergence(eps, p1, p2)
		if (is.infinite(c)) NA else c
	})
	names(n) <- c('p1', 'p2', 'n')
	print(plotConvergence(n, name))
	n
}



ab.epsilonFn <- function(p1, p2) {
	0.5
}

epsilon.epsilonFn <- function(p1, p2) {
	if (p1 > p2) 0.95 else 0.05
}

prob.epsilonFn <- function(p1, p2) {
	p1 / (p1 + p2)
}

hybrid.epsilonFn <- function(p1, p2) {
	epsExploit <- if (p1 >= p2) 1.0 else 0.0
	epsExplore <- 0.5
	epsProb <- p1 / (p1 + p2)
	alpha <- 0.1
	beta <- 0.1
	eps <- alpha * epsExploit + 
		beta * epsExplore + 
		(1 - (alpha + beta)) * epsProb
	eps
}

ab.convergence <- evaluateConvergence(
	ab.epsilonFn, 
	"Testare AB")
epsilon.convergence <- evaluateConvergence(
	epsilon.epsilonFn, 
	"Greedy Epsilon")
prob.convergence <- evaluateConvergence(
	prob.epsilonFn, 
	"Probabilistic Bandit")
hybrid.convergence <- evaluateConvergence(
	hybrid.epsilonFn, 
	"Hybrid Bandit")
	\end{lstlisting}
\end{center}

\section{Analiza regretului}

Folosind o parte din funcțiile expuse anterior, vom construi o serie de funcții utile pentru studiul regretului.

\begin{center}
	\begin{lstlisting}[language=r]
plotRegret <- function(data, title) {
	ggplot(data, aes(p1, p2, z = r)) +  
		stat_contour() + 
		geom_tile(aes(fill = r)) + 
		scale_fill_gradient(
			name = "Regretul mediu",
			limits = c(0, 1),
			low = "#ffffff",
			high = "#ff3333"
		) +
		labs(x = "Grup 1", y = "Grup 2") +		
		ggtitle(title)
}





evaluateRegret <- function(epsilon, name) {
	r <- mdply(data, function(p1, p2) {
		eps <- epsilon(p1, p2)
		if (p1 < p2) 
			eps * (p2 - p1) 
		else 
			(1 - eps) * abs(p1 - p2)
	})
	names(r) <- c('p1', 'p2', 'r')
	print(plotRegret(r, name))
	r
}

ab.regret <- evaluateRegret(
	ab.epsilonFn, 
	"Testare AB")
epsilon.regret <- evaluateRegret(
	epsilon.epsilonFn, 
	"Greedy Epsilon")
prob.regret <- evaluateRegret(
	prob.epsilonFn, 
	"Probabilistic Bandit")
hybrid.regret <- evaluateRegret(
	hybrid.epsilonFn, 
	"Hybrid Bandit")

	\end{lstlisting}
\end{center}

\end{appendices}

